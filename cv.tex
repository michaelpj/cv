\documentclass[12pt,a4paper,sans]{moderncv}
\moderncvstyle{banking}
\moderncvcolor{black}
\moderncvicons{awesome}

\usepackage[style=authoryear,sorting=ydnt,bibencoding=utf8]{biblatex}
\usepackage{lmodern}
\usepackage{fontawesome5}
\bibliography{cv}

\firstname{Michael}
\familyname{Peyton Jones}
\email{me@michaelpj.com}
\homepage{www.michaelpj.com}
\social[twitter]{mpeytonjones}
\social[linkedin]{michael-pj}
\social[github]{michaelpj}

\begin{document}

\maketitle

\begin{itemize}
  \item Experienced research engineer, with a track record of developing and implementing novel business-relevant research.
  \item Lead software developer, capable of managing the evolution of complex, business-critical projects.
  \item Reliable team leader, with experience facilitating a team of peers, and balancing the requirements of long-term research and short-term business needs.
  \item Lifelong learner, comfortable with and experienced at mastering new domains.
\end{itemize}

\section{Experience}
\cventry
{September 2021-Present}
{Technical Architect}
{Input Output HK}
{London}
{}
{
  Technical Architect for the Plutus team, working on smart contract support for the Cardano blockchain.
  \begin{itemize}
    \item Wrote and shepherded multiple popular design improvments to Cardano.
    \item Led the way in opening up the design of the system to the community through public design documents.
    \item Shifted the management of internal dependencies via versioned software in a package repository rather than an ad-hoc web of source pins.
    \item Author on two research papers.
    \item Acted as Product Owner, defining the team's goals and work.
    \item Continued technical work, including prototyping a major change to the Plutus Core language providing a ~30\% speedup.
  \end{itemize}
}

\cventry
{October 2019-September 2021}
{Software Engineering Lead}
{Input Output HK}
{London}
{}
{
  Team lead for the Plutus team, working on smart contract support for the Cardano blockchain.
  \begin{itemize}
    \item Shepherded the design of a complex, multi-year project from shortly after conception to integration into the Cardano mainnet.
    \item Managed team and product priorities, requirements, and roadmap.
    \item Grew the team, including splitting of a sister team to work on application suport.
    \item Pushed for and helped define new roles for the company, including Developer Experience engineers.
    \item Major author on three research papers.
    \item Acted as technical point of contact for partners and business development.
    \item Continued technical work, including speeding up the Plutus Core evaluator by ~40\%.
  \end{itemize}
}

\cventry
{July 2018-October 2019}
{Compiler Engineer}
{Input Output HK}
{London}
{}
{
  Compiler Engineer on the Plutus team, working on smart contract support for the Cardano blockchain.
  \begin{itemize}
    \item Implemented a Haskell to Plutus Core compiler as a GHC compiler plugin.
    \item Designed and implemented an intermediary language for Plutus Core.
    \item Maintained and improved developer experience for the team, including significant amounts of Nix work.
    \item Spoke and wrote about our work at meetups and conferences.
    \item Major author on two research papers.
  \end{itemize}
}

\cventry
{April 2013-July 2018}
{Research Engineer/Lead Software Developer}
{Semmle}
{Oxford}
{}
{
  Research Engineer and Lead for the QL team, working on the QL logic programming language, compiler, and toolchain.
  \begin{itemize}
    \item Programming language design for QL, including designing and implemented a semantics for recursive aggregates, which allowed efficient computation of additional graph algorithms in QL.
    \item Compiler and optimizer engineering, including implementing multi-stage caching of compilation and optimization results, which improved compilation speed by orders of magnitude in common cases.
    \item Leadership and management, including managing the implementation of parallel query evaluation, which was a six month project but delivered on time.
    \item Research and intellectual property development, including developing, implementing, and publishing original research on incremental computation.
    \item Offensive security research and variant analysis with QL.
    \item Customer success and sales engineering.
  \end{itemize}
}

\cventry
{June 2016-December 2017}
{Co-founder}
{Good Technology Project}
{Oxford}
{}
{
  This was a research project and consultancy focussed on social impact in technology and entrepreneurship. 
  Some examples of work that I did:
  \begin{itemize}
    \item Advised companies, institutions, and individual entrepreneurs on impact strategy.
    \item Developed qualitative and quantitative modelling techniques for assessing the potential impact of startups.
    \item Interviewed experts and conducted synoptic desk research to break down problem areas.
  \end{itemize}
}

\nocite{*}
\printbibliography[title={Publications}]

\section{Education}
\cventry{2008-2012}{Mathematics and Philosophy}{Oxford University}{Oxford}{}{MMathPhil - First Class}
\cventry{2011}{Jodrell Scholarship}{}{}{}{For performance in Final Examinations}

\section{Skills}

\subsection{Programming languages}

\cvitem{Fluent}{Haskell, Java, Nix}
\cvitem{Proficient}{Python, Scala}
\cvitem{Conversant}{Javascript, Typescript, Rust, C\#, C/C++, Idris}

\section{Interests}
I sing with Chantage in London, as well as occasionally returning to the Queen's College Chapel Choir. 
I like reading, swing dancing, and playing squash, when I get the opportunity. 
I am a member of Giving What We Can.

\end{document}
